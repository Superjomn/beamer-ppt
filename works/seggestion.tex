\documentclass[a4paper]{ctexart}
\usepackage{geometry}
\usepackage{ulem}
\geometry{left=2cm,right=2cm,top=2cm,bottom=2cm}
\usepackage{color}
\author{Chunwei Yan}
\title{二轮全书复习建议}
\begin{document}
    \maketitle
%---content here----
还有差不多两个月就到考试了,剩下来的时间,需要好好把握,等于就是考试前的准备工作。
\par
复习全书的二轮,需要高效率完成。数学考试的稳步发挥,需要对知识点和概念有透彻的理解,这些就是做全书和看教科书的意义所在。而知识点的稳固,只有在题目的应用中不断积累才行。
\par
计划好时间和任务,接下来一个月,好好把全书里面的知识点理清楚,多多练习。 对基础的方法题型要特别熟悉,同时能够快速计算出结果。
\par
要求方面,要注意,这一轮是考试前打基础的最后一轮,需要确保质量。在做题的过程中,一方面测试自己对相关基础知识的掌握程度,二来,查漏补缺,实在记不住,在书上打上记号,或者直接记录到笔记里面,天天有时间就看,就默写。

\section{考试的考察方面}
考试只考察三个方面:{\heiti \textcolor{red}{\textbf{对知识点的熟练程度、基础题型、计算能力。}}}
从这三个角度出发,有针对性地注意注意。 
\subsection{对知识点的熟练程度}
复习的时候,查漏补缺,珍惜每一道题,每道题都有它联系到的知识点框架,构建知识网络。
\subsection{基础题型}
大部分题目都是固定套路,在全书中做一皆题时,要努力去掌握它们的共性——题型。 记住套路的话,以后就会做了。
\subsection{快,准的计算能力}
千万不能手高眼低。 计算是数学题目完成的最后一个阶段,是把思想输出的质变的过程。就像一个自来水厂,如果连输出的管道都没有的话,有再多水都没有意义。
\par
和知识点类似,计算能力也蕴含在题目的踏实解答中。 当然,除非对一类题有特别大的把握可以略过,其他的情况一定要亲手计算到最后结果。
\par
这个计算不是简单看着题目的解答,自己一步一步验证一下。 是完全不看解答,独立做出最后结果的要求。
\par
同时,在计算正确的前提下,努力提高速度。 以实际考试的要求在日常的复习中提高计算能力。

\section{不骄不躁,从学习和知识积累的角度}
这个时间离考试已经不太远了,所以,从各种角度讲,这个阶段是考试前最关键也是最能够提分的阶段。
\par
但这个阶段没有必要紧张,都是过来人了。 平常心地走下去。 不要慌张,但前提是认真做好自己的事情。 少乱想,多做事就OK了。
\par
有的章节,可能觉得漏洞比较多,这个不需要紧张,这个阶段就是记忆知识点和题型的阶段,忘了,就再把这部分知识认真复习一下。必要的话,可以每天有时间就把自己没有把握的知识点练一练,看一看。
\par
以学习和吸收知识的角度来进行这一轮,遇到不熟悉的知识点,不要慌张。之所以需要复习,就是自己还没有完全懂,高效率地弄懂了。
\par
小的知识点可以做笔记,怎么方便怎么来,长篇大论就没有必要抄了,标记一下,多看看。

\section{一叶知秋,从一道题联系到一个知识网络}
\par
在练习的时候,一定要踏实,做一道题不容易,一定要把这道题中包含的知识点都弄透了,可以把它涉及到的相关的知识也看看,一叶知秋。可以高效地维护起一张知识网。
\par
比如,做到一道题,有用到三角函数的求导,如 $\sin'{x}=\cos{x}$,但就这个式子而言,自己可能知道。 但是如果就此止步,那么练习的效果就打折扣了,在考前查漏补缺的阶段,是否有必要把其他的三角函数求导的式子也看看,如比较复杂的$\arctan'{x}=?$,因为考试的时候肯定会考到三角函数的求导。从题目中的一个知识点联系到一个知识体系,常用的知识点多练习,那么考试肯定会比较熟悉了。
\par
知识点考试前肯定要弄透,如此,由一点联系面,能够很高效地掌握,同时联系起来也有助于记忆。公式可以抄写然后默写,发现有弄错的,做记号或者记录到笔记本里面,总之要特别对待。 
\par
不要觉得麻烦,知识点只能通过多看多练习。 一些公式自己觉得比较麻烦或者记不住,就是因为练习不多不熟练。 自己要创造机会多熟悉熟悉知识点。 数学三肯定是以基础知识为主要考察对象,而基础知识也是自己完全能够把握的。 
\par
如果某个知识点,自己早点掌握了,以后就不会花那么多时间去看了,同时碰到用到它的题目,自己也能够轻易解决了。



\section{数学复习有度,与其他科目协调提高}
数学有固定的时间,这个阶段没有必要瘸腿,时间肯定是够的,努力做事就好了。不要慌。
\par
其他科目也很重要,不要漠视,每一个科目都有需要达到的效果,要注意时间投入的合理。瘸腿是最不可取的。
\par
注意休息,张弛有度。 小的练习可以穿插进行,比如数学的公式默写,或者其他科目的小的知识点可以在零碎时间或者休息的时候看看,等于给大脑换个空气。


\section{结语}
宝贝,努力。 能够走到这个阶段,你做的很好。 但是,最后还需要继续努力,把平时的积累加强一下,量变转为质变,升华到考试的水平。
\par
缺啥补啥,咱的目标是,学习知识,同时最后把试考好。
\par
什么事情都经历过。 把身体弄好了,身体好了,复习才能连续复习好。
\par
\zihao{3}\heiti\textcolor{blue}{\textbf{宝贝加油,永远支持你,七色的岁月里,我们一起走!}}


\end{document}


