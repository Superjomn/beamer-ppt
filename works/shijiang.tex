\documentclass[a4paper]{ctexart}
\usepackage{geometry}
\geometry{left=2cm,right=2cm,top=2.5cm,bottom=2.5cm}

\author{Chunwei Yan}
\title{物体题目试讲}
\begin{document}
    \maketitle
%---content here----
\section{引入}
同学们好,我们下面看第7道题.
\par
第7道题是一道非常有趣的,有关浮力的一道题。我不知道同学们在做这道题的时候有没有想到类似的问题。
\par
大家知道之前全球变暖问题的很热,那么我们南北级有很多冰山对不对,那么我突然有一个疑问需要大家帮我解答一下。 如果全球变暖,那么我们的海平面会上升还是下降?
有谁知道吗? 直接说。 
\par
恩,我们很多同学都猜的很好,相信我们一起看了这道题,大家会有更加理论的解答。
\par
那么我们下面开始解决这道题. 解题第一步是什么? 对,大家说的很对,是审题。 我们在通读一遍题目之后,需要划出关键点,也就是题眼。\\
好,我请一位同学读一下题,好,就你.\\
。。。\\
好的,题目读完了,大家告诉我题眼,我写下来,大家在下面用笔着重划一下。

好, 果汁,对,当然,后面括号里面的也很关键(密度大于水),杯口相平,对,冰块融化。

好,我们找好了题眼,那么,接下来一步,需要通过题眼,挖掘题目蕴含的物理信息。

这个步骤还是比较关键的,确认哪些变量是已知的。我们解题的时候,需要弄清楚,哪些是已知的变量,哪些是未知的变量。

好,我们由第一个信息,得到什么? (写下式子),请一位同学告诉我。  哪位同学告诉我他们是什么关系? 对,说的非常好。
那么第二个题眼,杯口相平,告诉我得到什么? 大家好好想想。 (写下式子)
第三个题眼,当然就是问题了,融化后会得到什么。 看看选项,就是一个有关液面溢出还是下降的问题。

那么,我刚开始就说这是一道有关浮力的题目,为什么我这么说? 从哪儿看出来的? 大家猜。 直接从图上看出来的啊。 

扯到浮力,最重要的一个原理,有谁告诉我?OK,这位同学。 对,说的很好,就是漂浮的物体的质量就是排出的液体的质量。

那么,我们这里,就会得到什么? (写个公式), 冰块的质量=排出果汁的质量。

所以,







\end{document}


