\documentclass[a4paper]{ctexart}
\usepackage{geometry}
\geometry{left=2cm,right=2cm,top=2cm,bottom=2.5cm}

\author{严春伟}
\title{第一次小报告}

\begin{document}
    \maketitle
%---content here----
通过这一次小报告,我对Zigbee 协议有了初步的了解. 
\section{认识}
\par
Zigbee 是一个比局域网还要小的网络协议,可以说,它的出现,正是顺应了实际的需求,同时,也由于是更小的网络,是的zigbee有了跟多的灵活性。我们可以看到,它在很多方面有了很多的微调。
\par
首先,zigbee有用于中央仲裁的协调器,利用这样一个类似于中央管理的设备,zigbee简化了子设备的复杂度。如同电信网络一样,中央的中转设备比较强大,处理能力较强,使得边缘的电话机可以很简单。如此,简化了边缘设备的设计难度,简化了网络接入的接口的复杂度。
\par
zigbee利用超帧进行同步,利用CAP阶段进行竞争的传输,而在CFP阶段为一些预先设定的应用保留传输的位置,有效地协调了紧急应用和普通应用的关系,有很大的灵活性。

\section{对遗留问题的思考}
课堂上老师和同学们讨论了很多问题,我这边只列出我对其中两个问题的想法。

\subsection{超帧中Beacon不对齐}
Beacon在超帧中,承担着同步的作用。 而正常的同步,也许只需要几个bit的标志便足够了。 所以,此处Beacon肯定不可能占有整数个时槽。
\par
但是,为何后面CAP中,第0个时槽不对齐,我想是没有必要对齐了。 当然我不确信了解zigbee整个协议运行的具体细节。 我作如下假设,当zigbee设备接受到zigbee协调器发送的超帧后,立刻在开始的CAP阶段进行竞争。也许它们竞争的时候,可以在超帧指定的时槽中间,然后在下个整时槽的时候按照同步的安排发送数据。 因此,尽管发送的过程必须要有对齐的时槽,但是竞争的时候对时槽的对齐与否并没有特殊的要求。 因此,zigbee设备同样可以在非对齐的第0个时槽破碎的时间段里面,进行竞争的活动,然后在下一个合适的整时槽的时候发送数据便可。 如此能够充分利用时间。

\subsection{CAP和CFP对调用于紧急数据发送}
在第二个同学王惠玉同学的报告中,大家一起讨论了CAP和CFP对调后,对紧急数据发送带来的好处。
\par
在课堂上,我提出,对调后,有紧急数据要发送的应用,如果有大量数据需要发送,可以在CFP阶段固定的时槽发送数据后,还参加CAP阶段的竞争发送。 当时老师有提到发送的概率问题可能影响效率。 经过思考,我仍旧认为原来的想法是可行的。
\par
首先,CAP和CFP处在同一个超帧同步阶段内,如此,如果是紧急数据,尽管在CFP阶段有固定时槽用于发送,但是,如果数据量比较大,一次也许发送不完。那么,在随后的CAP阶段参与竞争也有一定的几率能够把余下的数据发送一部分出去。如此,发送的阶段可以按照概率分为两个,一个是100\%能够发送一部分的CFP阶段,一个是有一定几率能够发送一部分的CAP阶段,那么如果充分利用这两个阶段的话,大数据量的紧急数据肯定能够更加高效传输完。
\par
关键是注意,CAP和CFP在同一个周期内,如果紧急数据不珍惜CAP阶段,那么它必须要等到下一个周期才能够继续传输余下的数据。

\end{document}


