\documentclass[a4paper]{ctexart}
\usepackage{geometry}
\usepackage{pstricks}
\usepackage{multicol}
\usepackage{pst-plot}
\geometry{left=2cm,right=2cm,top=2.5cm,bottom=2.5cm}

\author{严春伟\\
    1201213679}
\title{2012科研岗位总结}
\begin{document}
    \maketitle
    \section{本学期安排的科研岗位任务和目标}
    学习Java,尝试运用Hadoop;阅读语言模型相关论文,争取将Google翻译的原理运用到文本摘要上。

    \section{自我评价完成情况}
    \begin{enumerate}
        \item 学习了Java,并用Java完成了彭波老师协同过滤的大作业
        \item 比较集中地看了一些有关统计机器翻译的论文,看了李航的统计学习方法,现在在看学长推荐的《Machine Learning》,用统计和机器学习的方法完成了推荐系统的大作业
        \item 在本机搭建了单机版的Hadoop,但还没有开始用Java编写其程序
    \end{enumerate}

    \section{周报月报}
    按时提交了月报,无迟交漏交情况。

    \section{完成成果情况}
    \subsection{阅读论文情况}
    \par 阅读论文列表:
    \begin{tabular}[]{l|l}
        <++>
    \end{tabular}<++>

    \section{动手情况}
    \subsection{完成代码量}

    \subsubsection {Swin-Info-Center}
    \par 密码学大作业 
    \begin{tabular}[]{l|l}
    \hline
    语言            &   Python, Html, CSS, Javascript   \\
    \hline
    Python代码数    &   2780行                          \\
    \hline
    Html            &   1347行                          \\
    \hline
    CSS             &   495行                           \\
    \hline
    \end{tabular}

    \subsection{yelldog}
    \par 网络体系结构协同过滤大作业
    \begin{tabular}[]{ll}
        \hline
        语言        &   Java                            \\
    \hline
        Java        &   671行                           \\
    \hline
    \end{tabular}

    \subsection{redog}
    \par 网络体系结构推荐系统大作业
    \begin{tabular}[]{l|l}
        \hline
        语言        &   C++                             \\
        \hline
        C++         &   1775行                          \\
        \hline
    \end{tabular}

    \subsection{TDA}
    \par 摸板识别大作业
    \begin{tabular}[]{l|l}
        \hline
        语言        &   Python                             \\
        \hline
        Python          &   1891行                          \\
        \hline
    \end{tabular}



    
    


\end{document}

