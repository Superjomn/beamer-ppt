\documentclass{beamer}
\usepackage{ctex} %注意这个宏包
\usepackage{multicol}
\usepackage{color}
%\usetheme{umbc4}
\usetheme[height=12mm]{Rochester}
\usetheme{Berkeley}
\usepackage{pst-plot}
\CTEXoptions[today=old]
\title{微博followee 推荐}
\author{Chunwei Yan}
\institute[PKUSZ]{
\texttt{superjom@sz.pku.edu.cn}
}
\date{\today}

\begin{document}
% ------------- title page ----------------------------
%--- the titlepage frame -------------------------%
\begin{frame}[plain]
  \titlepage
\end{frame}

% ------------- question ---------------------
\begin{frame}{背景}
    \begin{itemize}
    \item 社交网络不断发展
        \begin{itemize}
        \item Twitter
        \item Facebook
        \item Renren
        \item MicroMsg
        \end{itemize}
    \item 信息过载
        \begin{itemize}
            \item 感兴趣的信息
            \item 感兴趣的人
        \end{itemize}
    \item  推荐系统
        \begin{itemize}
        \item 根据用户的已有信息,建立用户的兴趣模型
        \item 推测用户感兴趣的,推荐
        \end{itemize}
    \end{itemize}
\end{frame}

\begin{frame}{KDDCUP 2012 Track1}
    \begin{itemize}
    \item 目标: 推测一个用户是否会follow 推荐的好友
    \item datasets: 
        \begin{enumerate}
        \item 训练集:(UserId)(ItemId)(Result)(Unix-timestamp)
        \item 其他信息:
            \begin{itemize}
                \item Profile: (UserId)(Year-of-birth)(Gender)(Number-of-tweet)(Tag-Ids)
                \item item: (ItemId)(Item-Category)(Item-Keyword)
                \item user-action: (UserId)(Action-Destination-UserId)(Number-of-at-action)(Number-of-retweet)(Number-of-comment)
            \end{itemize}
        \end{enumerate}
    \end{itemize}
\end{frame}

\begin{frame}{相关工作}
    \begin{block}{上海交大}
        \begin{itemize}
        \item 1st-0.4265
        \item Combining Factorization Model and Additive Forest for Collaborative Followee Recommendation
        \end{itemize}
    \end{block}

    \begin{block}{盛大研究院}
        \begin{itemize}
        \item 2st 
        \item Context-aware Ensemble of Multifaceted Factorization Models for Recommendation Prediction in Social Networks
        \end{itemize}
    \end{block}
\end{frame}

\begin{frame}{方法总结}
    \begin{block}{SVD 矩阵分解}
    \end{block}
    
    \begin{block}{Additive Forest}
    \end{block}

    \begin{block}{排序学习(Learn to Rank)}
    \end{block}
\end{frame}

\begin{frame}{处理流程}
    \begin{enumerate}
    \item 数据分析及预处理
    \item 建立模型
    \item 多模型参数学习
    \item 合并模型
    \end{enumerate}
\end{frame}
% 开始各个步骤的讲解 ---------------------------------------
% ----------------------------------------------------------
\begin{frame}{1. 数据分析及预处理}
    \begin{itemize}
    \item 日期时间
        \begin{itemize}
        \item 日期,星期
        \item 一天中早晚时间
        \end{itemize}
    \item 年龄
    \end{itemize}
\end{frame}

\begin{frame}{2. 建立模型}
    \begin{itemize}
    \item SVD / SVD++
    \item Addative Forest
    \item KNN
    \end{itemize}
\end{frame}

\begin{frame}{3. 多模型参数学习}
\end{frame}

\begin{frame}{4. 合并模型}
\end{frame}


\end{document}
