\documentclass{beamer}
\usepackage{ctex} %注意这个宏包
\usepackage{multicol}
\usepackage{color}
\usepackage{graphics,graphicx}
\usepackage{pstricks,pst-node,pst-tree}
%\usetheme{umbc4}
%\usetheme[height=12mm]{Rochester}
%\usetheme{Marburg}
\usetheme{Berkeley}
\useinnertheme{umbcboxes}
\useinnertheme{rounded}
\usepackage{pstricks}
\usepackage{pst-plot}
\CTEXoptions[today=old]
%\setbeamercolor{beaver}{bg=violet!15,fg=black}  % redefine box color!

\title{Reliable Optimised Flooding in Ad hoc Networks}
\author{Chunwei Yan}
\institute[PKUSZ]{
\texttt{YanChunwei@outlook.com}
}
\date{\today}

\begin{document}
\section{Begin}
% ------------- title page ----------------------------
%--- the titlepage frame -------------------------%
\begin{frame}
  \titlepage
\end{frame}

\subsection{Abstract}
\begin{frame}{Abstract}

\end{frame}
\subsection{Basic Knowledge}

\begin{frame}{Ad hoc Networks}
\begin{definition}
Ad hoc网是一种多跳的、无中心的、自组织无线网络,又称为多跳网(Multi-hop Network)、无基础设施网(Infrastructureless Network)或自组织网(Self-or-ganizing Network)。
\end{definition}
\begin{enumerate}
    \item 整个网络没有固定的基础设施,每个节点都是移动的
    \item 由于终端无线覆盖取值范围的有限性,两个无法直接进行通信的用户终端可以借助其它节点进行分组转发。
    \item 每一个节点同时是一个路由器,它们能完成发现以及维持到其它节点路由的功能。 
\end{enumerate}
\end{frame}

\begin{frame}{Ad hoc Networks}
    \begin{center}
    \includegraphics[height=140pt]{ad_hoc_network.jpg}
    \end{center}
\end{frame}

\begin{frame}{Unicast(单播)}
    \begin{definition}
        \begin{itemize}
            \item a piece of information is sent from one point to another point. 
            \item just one sender, and one receiver.
        \end{itemize}
    \end{definition}
    $
    \psmatrix[colsep=2cm,rowsep=1cm,mnode=circle]
    1&2
    \ncline{->}{1,1}{1,2}
    \endpsmatrix
    $
\end{frame}

\begin{frame}{Broadcast(广播)}
    \begin{definition}
        \begin{itemize}
            \item a piece of information is sent from one point to all other points. 
            \item just one sender, but the information is sent to all connected receivers.
        \end{itemize}
    \end{definition}
    $
    \psmatrix[colsep=2cm,rowsep=1cm,mnode=circle]
    1&2\\
    &4&5\\
    6&7\\
    \ncline{->}{1,1}{1,2}
    \ncline{->}{1,1}{2,2}
    \ncline{->}{1,1}{2,3}
    \ncline{->}{1,1}{3,1}
    \ncline{->}{1,1}{3,2}
    \endpsmatrix
    $

\end{frame}


\subsection{subsection}

\end{document}
