\documentclass{beamer}
\usepackage{ctex} %注意这个宏包
\usepackage{multicol}
\usepackage{color}
\usepackage{graphics,graphicx}
\usepackage{pstricks,pst-node,pst-tree}
\usepackage{listings}
%\usetheme{umbc4}
%\usetheme[height=12mm]{Rochester}
%\usetheme{Marburg}
\usetheme{Berkeley}
\useinnertheme{umbcboxes}
\useinnertheme{rounded}
\usepackage{pstricks}
\usepackage{pst-plot}
\CTEXoptions[today=old]
%\setbeamercolor{beaver}{bg=violet!15,fg=black}  % redefine box color!

\title{Reliable Optimised Flooding in Ad hoc Networks}
\author{Chunwei Yan}
\institute[PKUSZ]{
\texttt{YanChunwei@outlook.com}
}
\date{\today}

\begin{document}
% ------------- title page ----------------------------
%--- the titlepage frame -------------------------%
\begin{frame}
  \titlepage
\end{frame}

\section{Begin}
\begin{frame}
\frametitle{Outline}
\tableofcontents
\end{frame}


\subsection{Abstract}
\begin{frame}{Flooding}
    \begin{block}{Blind Flooding}
        \begin{itemize}
        \pause
        \item remarkably robust and is able to reliably deliver messages
        \pause
        \item results in the broadcast storm problem
        \end{itemize}
    \end{block}

        \pause
    \begin{block}{Optimised Flooding}
        \begin{itemize}
        \pause
        \item limits the broadcast storm problem
        \pause
        \item reduces the inherent level of redundancy
        \end{itemize}
    \end{block}
\end{frame}
\begin{frame}{RMST Flooding}
    \begin{block}{RMST--Reliable Minimum Spanning Tree}
        \begin{itemize}
        \pause
        \item replace broadcast transmissions with unicast transmissions
        \pause
            \begin{itemize}
            \item link layer acknowledgement and retransmission
            \end{itemize}
        \item improve the reliability of a flood
        \pause
        \item reduce the broadcast storm problem
        \end{itemize}
    \end{block}
\end{frame}

\subsection{Basic Knowledge}
\begin{frame}{Ad hoc Networks}
        \pause
\begin{definition}
An ad hoc network is a collection of wireless mobile nodes forming a temporary network lacking tranditional centralised administration
\end{definition}
        \pause
\begin{enumerate}
    \item 不固定,移动性
        \pause
    \item 相互协作,远距离传递转发
        \pause
    \item 每一个节点同时是一个路由器
\end{enumerate}
\end{frame}

\begin{frame}{Ad hoc Networks}
    \begin{center}
    \includegraphics[height=140pt]{ad_hoc_network.jpg}
    \end{center}
\end{frame}

\begin{frame}{Unicast(单播)}
        \pause
    \begin{definition}
        \begin{itemize}
            \item a piece of information is sent from one point to another point. 
            \item just one sender, and one receiver.
        \end{itemize}
    \end{definition}
        \pause
    $
    \psmatrix[colsep=2cm,rowsep=1cm,mnode=circle]
    1&2
    \ncline{->}{1,1}{1,2}
    \endpsmatrix
    $
\end{frame}

\begin{frame}{Broadcast(广播)}
        \pause
    \begin{definition}
        \begin{itemize}
            \item a piece of information is sent from one point to all other points. 
        \pause
            \item just one sender, but the information is sent to all connected receivers.
        \end{itemize}
    \end{definition}
        \pause
    $
    \psmatrix[colsep=2cm,rowsep=1cm,mnode=circle]
    1&2\\
    &4&5\\
    6&7\\
    \ncline{->}{1,1}{1,2}
    \ncline{->}{1,1}{2,2}
    \ncline{->}{1,1}{2,3}
    \ncline{->}{1,1}{3,1}
    \ncline{->}{1,1}{3,2}
    \endpsmatrix
    $
\end{frame}

\section{Proposed Reliable Flooding Mechanism}
\begin{frame}{Broadcast Flood Problem}
\includegraphics[height=130pt]{broadcast_flood_problem.png}
\end{frame}

\subsection{RMST}
\begin{frame}{RMST--Reliable Minimum Spanning Tree}
RMST is a reliable and optimised flooding mechanism taht computes a local MST based upon one hop neighbour knowledge in a distributed manner.
\end{frame}
\begin{frame}{RMST}
        \pause
\begin{center}
    \includegraphics[height=100pt]{rmst.png}
\end{center}
    \begin{itemize}
        \pause
    \item the centralised MST $\sqsubseteq$ distributed MST 
        \pause
    \item the distributed MST results in a connected graph with 
        \begin{itemize}
        \pause
        \item a neighbour degree greater than one but less than six 
        \pause
        \item an average neighbour degree of less than 2.04 nodes
    \end{itemize}
    \end{itemize}
\end{frame}

\begin{frame}{Algorithm}
Algorithm RMST(message)\\
    \quad if not seen message before\\
            \quad\quad BSET <- MST(!-hop Neighbours)\\
            \quad\quad i <- previous broadcasting node\\
            \quad\quad H <- nodes that recieved previous broadcast\\
            \quad\quad BSET <- BSET - i\\
            \quad\quad BSET <- BSET - H\\
            \quad\quad for each node i in BSET\\
                \quad\quad \quad\quad $T_{power}$ <- $transmission_{power}(i)$\\
                \quad\quad \quad\quad $Unicast(Message, T_{power})$\\

\end{frame}

\subsection{Unicast Transmission Mechanism}

\begin{frame}{Unicast Transmission Mechanism}
\begin{center}
    \includegraphics[height=80pt]{arq.png}
\end{center}
        \pause
\begin{itemize}
\item unicast transmission utilises a frame retransmission mechanism at the MAC layer
        \pause
\item positive acknowledge ment scheme(ARQ)
        \pause
\item a transmitting node will retransmit a frame if it does not receive a positive acknowledgement from the destination node
\end{itemize}
\end{frame}


\section{Conclusions}
\subsection{Broadcast Reachability}
\begin{frame}{Broadcast Reachability}
    \begin{center}
    \includegraphics[height=100pt]{f1.png}
    \end{center}
    bind flooding and RMST provide the best delivery performance and are only slightly affected by background traffic.
\end{frame}

\subsection{Energy consumed}
\begin{frame}{Energy consumed}
    \begin{center}
    \includegraphics[height=100pt]{f2.png}
    \end{center}
    It shows the power consumed by each mechanism to complete a flood.\\
    RMST utilises more energy to complete a flood than LMSTflood.
\end{frame}

\begin{frame}{Goodbye}
    \zihao{0}
    Tankyou!
\end{frame}

\end{document}
