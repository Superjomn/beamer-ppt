\documentclass[a4paper]{ctexart}
\usepackage{geometry}
\geometry{left=2cm,right=2cm,top=2.5cm,bottom=2.5cm}

\author{Chunwei Yan}
\title{Homework 1}
\begin{document}
    \maketitle
%---content here----
\section{大体思路}
原题中限制了直接获取全局图的能力,而是只单次提供每一个节点及其neighbour的信息。\\
如此,想到从微观角度进行计算。 即,每次只计算一个节点的pagerank值,并且不断在磁盘中进行覆盖重写。 全局利用广度优先的方式在neighbour中遍历开来。当然,也需要考虑BFS对孤立集合的问题。每次BFS终止时,随机从未计算的节点里面选取一个节点,继续计算用BFS进行计算。 \\
初始时,同样为每一个节点分配$\frac{1}{N}$的初始值,然后按照以下公式计算
$$
P_i = k\sum_{P_j \j Neibours}{\frac{P_j}{L_j}} + (1-k)\frac{1}{N}
$$
\section{算法解释}
就像一滴水落到水面,刚开始只是一点有很大的起伏,之后水面会溅起涟漪,不停震荡,最终又会恢复到平面。 那多余的一滴水在不停震荡之后,自然会平稳地分布到平面。\\
用初始时从一点(可能也可以从多点)计算来模拟一滴水掉到水平面,改进的BFS来模拟涟漪的扩展,用不停迭代代表震荡,而最终平静之后,就是各点的pagerank值的稳定。\\
我们尝试用不停的微观的微调来模拟宏观的调整。\\
\section{一些注意点}
\begin{enumerate}
    \item 注意磁盘与内存中内容的同步
    \item BFS终止时,需要从此轮未计算过的节点中选取一个,继续BFS
    \item 每一次计算的节点,都需要是此轮未计算过的节点,或许在磁盘中需要保持一个标记
\end{enumerate}
\end{document}


