\documentclass[a4paper]{ctexart}
\usepackage{geometry}
\usepackage{pstricks}
\usepackage{multicol}
\usepackage{pst-plot}
\geometry{left=2cm,right=2cm,top=2.5cm,bottom=2.5cm}

\author{严春伟}
\title{数字城市与网络信息安全}
\begin{document}
    \maketitle

    \section{数字城市发展概况}
    \par 数字城市是以计算机技术、多媒体技术和大规模存储技术为基础,以宽带网络为纽带,运用遥感、全球定位系统、地理信息系统、遥测、仿真-虚拟等技术,对城市进行多分辨率、多尺度、多时空和多种类的三维描述,即利用信息技术手段把城市的过去、现状和未来的全部内容在网络上进行数字化虚拟实现。 \cite{china-digital-city}

    \section{数字城市支撑技术}
    \subsection{遥感技术}
    \par 数字城市致力于将城市数字化建模.在城市数字化方面,遥感技术有很大的应用空间.
    \begin{enumerate}
        \item 拓展综合信息源. 现代城市中有极丰富的人文社会化信息,利用GIS和遥感集成技术将此方面信息整合起来,为宏观上探测城市的综合信息提供了技术可能.
        \item 动态变化. 遥感技术可以动态更新,实现数据实时性,趋势性的要求.
        \item 监测评估功能. 能够对城市资源,经济,人文实时监控. 通过反馈评估,使城市健康高效运行.
    \end{enumerate}

    \subsection{宽带网络}
    \par 信息技术的发展能实现信息空间的扩展与城市空间的延伸.
    \begin{enumerate}
        \item 通过信息技术的支持,能集中城市经济规划与管理的职能,提供生产和消费市场的在线服务,消除信息互动的物质空间距离.
        \item 提供便利的城市社会文化生活,为市民提供相关的服务. 数字城市的管理中心能够便捷地与市民间实现便捷而方便的互动.
        \item 促进虚拟社区的发展. 强化了物质社区的功能,推动信息时代社区功能的全面复兴和发展.
    \end{enumerate}

    \subsection{网络技术}

    \section{数字城市中网络信息安全的必要性}
    \section{网络信息安全相关技术支撑}
    \section{结论}
    \begin{thebibliography}{sotief}
            \bibitem{china-digital-city} 王家耀,宁津生,张祖勋.中国数字城市建设方案及推进战略研究.北京:科学出版社.2008
            \bibitem{digital-technology}许奕锋,试论数字城市建设与管理的三大技术支撑,湖南省委党校,湖南长沙
            \bibitem{security-technology} 沈昌祥 张焕国 冯登国 曹珍富 黄继武, 信息安全综述 ,中国科学,2007
    \end{thebibliography}
\end{document}

