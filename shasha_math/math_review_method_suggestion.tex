\documentclass[a4paper]{ctexart}
\usepackage{geometry}
\geometry{left=2cm,right=2cm,top=2.5cm,bottom=2.5cm}

\author{Chunwei Yan}
\title{全书复习方法建议}
\begin{document}
    \maketitle
%---content here----
\section{考前全书复习的关键}
    现在做题,是{\heiti 精练}。 \\
    不追求题量,做一道题是一道题.该练习的,该吸收的成分吸收够了,肯定比粗放地做N题都要高效的多。\\
    数学的各个地方原理都是相通的,多多把思维的触角拓展、各个章节知识相互联系,肯定能够记忆的很牢靠。\\
    其中,需要锻炼的能力主要有三点:

    \subsection{计算能力}
    把这个列在第一个,就是要加强重视。 之前或许没有专门加强,最后的这个阶段一定要注意。\\
    数学千万不要靠看,那个很不可靠。 不管是什么题目,最后总需要算出来才能够得分,如何在考试的时候,能够又快又准地计算出来,这个就是现在需要考虑和重点加强的地方.\\
    平时可以重点进行一些练习,像考试一样,限定时间自己独立做出来。\\
    也许时间比较紧,但是这个步骤绝对不能少,题目比较多的话,可以做个取舍,在保证计算、知识点、题型的吸收到位的基础之上,稍微加快速度。\\
    初始的时候,也许计算有点吃力,需要坚持。 坚持到最后,每天能够保证有一定量的练习就很好了,自然而然地计算能力就会有提高了。\\
    算错了没有关系,总结,查出错误的地方。但是计算这个步骤不能给少。\\

    \subsection{知识点}
    知识点是基础,可以参照{\heiti 考纲},把每一个知识点好好落实好。\\
    一道题目里面可能涉及到很多知识点,当然,如果能够自己独立计算出来,那应该知识点都比较熟悉。 
    如果是精练的话,题目涉及到的知识点,可以把知识点所联系到的同类的知识点也落实一下,比如,题目涉及到一个三角函数公式$\sin'{x}=\cos{x}$,可以把同类的三角函数都看一看,全书上都列出来了。这样由点到面,效率高很多。 不要担心多花时间
    \begin{itemize}
        \item 如果自己比较熟悉,那么也就是多扫一眼,如果特别熟悉了,都知道自己有没有必要看
        \item 如果自己比较生疏,那么肯定要多看一看,多练一练。知识点都是连通的,以后肯定会有很多题目会涉及到类似的知识点,自己能够做对一个系列,而不是仅仅针对一道题
    \end{itemize}
    把要考的知识点练熟练了,那么就能够有至少$\frac{1}{3}$的把握了。
    \paragraph{练习的方法}
    确定一点,所有知识点,考试之前都需要很熟练。不熟悉的知识点,不管怎么复杂,每天{\heiti 多抄写,多练练,多默写},过几天总会熟练起来的。\\
    硬骨头,重点应付,每天都练,练到熟练的那一天。
    \begin{itemize}
        \item 全书和我之前给你的那个公式小册子里面,几乎所有的知识点上面都整理好了。自己平时多看看
        \item 全书中,有题目涉及到相应的知识点,如果比较模糊的话,一定要回头好好看一看,具体参考一下,重温一下,这样能够保证前后相通,不至于忘.
        \item 如果有知识点不熟悉,做上标记,重点对待,不记得就天天看.
    \end{itemize}

    \subsection{题型}
    题型就是套路,这个套路的获得,需要做一定的题目,并且一定要进行归类(全书上题型应该比较具体)。\\
    重点还是计算——要自己独立计算出来。计算就是一个加强逻辑和深度记忆的过程。 \\
    数学不同于其他科目的一点,就是记忆是潜移默化的,只要你能够把这道题目计算出来,那么最重要也是最难的逻辑过程肯定就记住了,但是需要长期或者高效记忆的话,肯定需要后期的整理和分析。\\
    做完一道题,花点时间好好想想,精妙在哪里,跟前面那道题比较相似,在题目旁边记录一些感想或者注意点。 这一点很重要,就是一个帮助记忆和题型意识加强的过程。\\

\section{复习计划}
    之前是复习的时候按照章节,前后周期比较长,复习到后面前面就忘了。\\
    现在高数和线代和概率{\heiti 穿插着进行}很好。作为最后阶段的复习,注意前后内容记忆。\\
    当然,数学的独特之处,就是知识点都是想通的,所以,在后面的题目涉及到以前的知识点的时候,如果感觉有遗忘的话,回头好好再把那些知识点复习一下,问题就不大了。\\
    这个阶段的目标就是打基础,把计算能力、题型意识、知识点积累等方面达到考试的要求。\\
    要抱着将要去考试,去检验自己掌握程度的意识去复习,如果一个月后去考试,自己掌握的各方面还有欠缺吗? 那么现在需要加强的就是那些欠缺的方面。\\
    其实这个阶段的复习已经比较关键了,相当于一只脚已经踏上了起跑线了,所以,保证效率,在快考试的这个阶段,为考试做好准备。\\
    平时闲暇时间,可以看一两道真题(基础题,不是特别难的那种),体验一下真题的要求,自己宏观上做好准备。但是,不要专门花时间去看,这个阶段还没有必要。全书上的题型是真题的扩展,两者是想通的。\\

\section{有关身体和调节}
    身体是很关键的,还是要注意调节。 复习、休息、锻炼做好权衡。\\
    一般,冬天的时候总容易生个小病,所以自己注意着。\\
    \paragraph{有张有弛,不要太疲劳了哦}
    \begin{itemize}
        \item 中午和晚上吃饭的时候,多走走路,就当锻炼一下吧
        \item 紧张的时候,适当调节一下,不要过度 ,可以出去散散步,散步是一种很好的放松和深层次思考的过程,自己有什么麻烦事,可以散步的时候胡思乱想下,舒展下心情
        \item 蔬菜、水果,充足的VC保证,坚果什么的适度,上火的东西也少碰,特别是生理期的时候,衣食住行都注意点,毕竟不是平时
    \end{itemize}
    还是那句话,复习阶段,缺啥补啥,笨鸟先飞。带着方向感,朝着目标,日积月累,朝着目的地远航吧!\\
    现在目的地就在前方,这是挑战,更多也是机遇。 高效而充分地利用最后这个阶段,比别人做的好一点,肯定会有很大优势的!\\
    \zihao{2} {\heiti 莎莎加油哦,我永远支持你, 全力去追随梦想吧!}
    
\end{document}


