\documentclass[a4paper]{ctexart}
\author{Chunwei Yan}
\usepackage{geometry}
\geometry{left=2cm,right=2cm,top=2.5cm,bottom=2.5cm}

\title{极限和积分部分公式加强}
\usepackage{color}
\usepackage{graphicx}
\usepackage{multicol}

\begin{document}
    \maketitle
%---------------------- start document -------------------

\section{极限}

\subsection{几个重要的等价无穷小}
\paragraph{$x \rightarrow 0$}

%--table
\begin{center}
\renewcommand\arraystretch{1.5}
    \begin{tabular}{  l  l  l  l }
        %----------
        $\sin{x}\sim$ & 
        \underline{\hbox to 30mm{}} &
        $(1+x)^\frac{1}{x}=$ &
        \underline{\hbox to 30mm{}} \\
        %----------
        $\sin{x} \sim$  &
        \underline{\hbox to 30mm{}} &
        $\tan{x} \sim$   &
        \underline{\hbox to 30mm{}} \\
        %----------
        $1 - \cos{x} \sim $         &
        \underline{\hbox to 30mm{}} &
        $e^x - 1 \sim$              &
        \underline{\hbox to 30mm{}} \\
        $\ln{1+x} \sim$             &
        \underline{\hbox to 30mm{}} &
        $(1+x)^a - 1 \sim$          &
        \underline{\hbox to 30mm{}} \\
        $\arcsin{x} \sim$           &
        \underline{\hbox to 30mm{}} &
        $\arctan{x} \sim$           &
        \underline{\hbox to 30mm{}} \\
        $a^x - 1 \sim$              &
        \underline{\hbox to 30mm{}} &
        $x^m + x^k(k>m>0) \sim$       &
        \underline{\hbox to 30mm{}} \\
    \end{tabular}
\end{center}

\subparagraph{几个极限}
\begin{center}
\renewcommand\arraystretch{1.5}
    \begin{tabular}{  l  l  l  l }
    %-------
    $\lim_{n \rightarrow \infty} { \sqrt[n]{n}} =$   &
        \underline{\hbox to 30mm{}}                  &
    $\lim_{x \rightarrow 0^+}
          { x^\delta(\ln{x})^k }
        =$                                           &
        \underline{\hbox to 30mm{}} \\
    $\lim_{x \rightarrow +\infty}
        {
            x^k e^{-\delta x}
        }$                                           &
        \underline{\hbox to 30mm{}}                  & 
        (常数 $k>0$ \quad $\delta >0$)               &
        \\
    \end{tabular}
\end{center}


\section{初等函数的导数公式}
\textcolor{red}{\textbf{复习全书P41 \quad 公式2}}

\begin{center}
\renewcommand\arraystretch{1.5}
    \begin{tabular}{  l  l  l  l }
        $C'=$                           &
        \underline{\hbox to 30mm{}}     & 
        $\int{0dx}=$                    &
        \underline{\hbox to 30mm{}}     \\


        $(x^a)'=$                       &
        \underline{\hbox to 30mm{}}     & 
        $\int{ ax^{a-1} dx}=$           &
        \underline{\hbox to 30mm{}}     \\

        $(a^x)'=$                        &
        \underline{\hbox to 30mm{}}     & 
        $\int{ a^x\ln{a}dx}=$            &
        \underline{\hbox to 30mm{}}     \\

        $(e^x)'=$                       &
        \underline{\hbox to 30mm{}}     & 
        $\int{e^x dx}=$                  &
        \underline{\hbox to 30mm{}}     \\

        $(\log{a}{x})'=$                &
        \underline{\hbox to 30mm{}}     & 
        $\int{\frac{1}{x\ln{a}}dx}=$    &
        \underline{\hbox to 30mm{}}     \\

        $(\ln{x})'=$                    &
        \underline{\hbox to 30mm{}}     &
        $\int{\frac{1}{x}dx}=$          &
        \underline{\hbox to 30mm{}}     \\


        %--三角函数
        $(\sin{x})'=$                   &
        \underline{\hbox to 30mm{}}     & 
        $\int{\cos{x}dx}=$              &
        \underline{\hbox to 30mm{}}     \\

        $(\cos{x})'=$                   &
        \underline{\hbox to 30mm{}}     & 
        $\int{-\sin{x}dx}=$             &
        \underline{\hbox to 30mm{}}     \\

        $(\tan{x})'=$                   &
        \underline{\hbox to 30mm{}}     & 
        $\int{\sec{x}^2dx} = $          &
        \underline{\hbox to 30mm{}}     \\

        ${\cot{x}}'=$                   &
        \underline{\hbox to 30mm{}}     & 
        $-\int{\csc{x}^2dx}=$           &
        \underline{\hbox to 30mm{}}     \\

        $(\sec{x})'=$                   &
        \underline{\hbox to 30mm{}}     & 
        $\int{\sec{x}\tan{x}dx}=$       &
        \underline{\hbox to 30mm{}}     \\

        $(\csc{x})'=$                   &
        \underline{\hbox to 30mm{}}     & 
        $-\int{\csc{x}\cot{x}dx}=$      &
        \underline{\hbox to 30mm{}}     \\

        $(\arcsin{x})'=$                &
        \underline{\hbox to 30mm{}}     & 
        $\int{\frac{1}{\sqrt{1-x^2}}dx}=$&
        \underline{\hbox to 30mm{}}     \\

        $(\arccos{x})'=$                &
        \underline{\hbox to 30mm{}}     & 
        $-\int{\frac{1}{\sqrt1-x^2}dx}=$&
        \underline{\hbox to 30mm{}}     \\

        $(\arctan{x})'=$                &
        \underline{\hbox to 30mm{}}     & 
        $\int{\frac{1}{1+x^2}dx}=$      &
        \underline{\hbox to 30mm{}}     \\

        $(arccotx)'=$                &
        \underline{\hbox to 30mm{}}     & 
        $-\int{\frac{1}{1+x^2}dx}=$     &
        \underline{\hbox to 30mm{}}     \\

        $\int{\frac{1}{\cos^2{x}}dx}=$  &
        \underline{\hbox to 30mm{}}     & 
        $-\int{\frac{1}{\sin^2{x}}dx}=$ &
        \underline{\hbox to 30mm{}}     \\

    \end{tabular}
\end{center}


\subsection{几个常见初等函数的n阶导数公式}
    \textcolor{red}{\textbf{复习全书P42 \quad (5)}}
\begin{multicols}{2}
    $(e^{ax})^{(n)}=$
        \underline{\hbox to 30mm{}}     \\
    $(\sin{ax})^{(n)}=$
        \underline{\hbox to 30mm{}}     \\
    $(\cos{ax})^{(n)}=$
        \underline{\hbox to 30mm{}}     \\
    $\ln^{(n)}{(1+x)}=$
        \underline{\hbox to 30mm{}}     \\
    $((1+x)^\alpha)^{(n)}=$
        \underline{\hbox to 30mm{}}     \\
\end{multicols}

\subsection{幂指函数$u(x)^{v(x)}$的求导法则与公式}
    \textcolor{red}{\textbf{复习全书P42 \quad (7)}}\\
    $u(x)^{v(x)}$
        \underline{\hbox to 30mm{}}     \\

\subsection{反函数的一阶及二阶导数公式}
    \textcolor{red}{\textbf{复习全书P42 \quad (8)}}\\
设$y=f(x)$可导,且$f'(x)\neq 0$, 则存在反函数 $x=\varphi (y)$, 且\\
$$
\frac{dx}{dy} = 
    \frac{1}{
        \frac{dy}{dx}
        } $$
即\\
$$
\varphi'(y) = \frac{1}{f'(x)}
$$
若$y=f(x)$存在二阶导数,则\\
$ \varphi''(y) = $
        \underline{\hbox to 30mm{}}     \\

\section{不定积分与定积分的计算}
\subsection{基本积分公式}

\begin{center}
\renewcommand\arraystretch{1.5}
    \begin{tabular}{  l  l  l  l }
    %------------
    $\int{x^\alpha dx}=$        &
    \underline{\hbox to 30mm{}} &
    $\int{\frac{1}{x}dx}=$      &
    \underline{\hbox to 30mm{}} \\
    %------------
    $\int{a^xdx}=$              &
    \underline{\hbox to 30mm{}} &
    $\int{e^xdx}=$              &
    \underline{\hbox to 30mm{}} \\
    %------------
    $\int{\sin{x}dx}=$          &
    \underline{\hbox to 30mm{}} &
    $\int{\cos{x}dx}=$          &
    \underline{\hbox to 30mm{}} \\
    %------------
    $\int{\tan{x}dx}=$          &
    \underline{\hbox to 30mm{}} &
    $\int{\cot{x}dx}=$          &
    \underline{\hbox to 30mm{}} \\
    %------------
    $\int{\sec{x}dx}=$          &
    \underline{\hbox to 30mm{}} &
    $\int{\csc{x}dx}=$          &
    \underline{\hbox to 30mm{}} \\
    %------------
    $\int{\sec^2{x}dx}=$        &
    \underline{\hbox to 30mm{}} &
    $\int{\csc^2{x}dx}=$        &
    \underline{\hbox to 30mm{}} \\
    %------------
    $\int{\frac{1}{a^2+x^2}dx}=$&
    \underline{\hbox to 30mm{}} &
    $\int{\frac{1}{a^2-x^2}dx}=$&
    \underline{\hbox to 30mm{}} \\
    %------------
    $\int{\frac{1}{\sqrt{a^2-x^2}}dx}=$&
    \underline{\hbox to 30mm{}} &
    $\int{\frac{1}{\sqrt{x^2 \pm a^2}}dx}=$&
    \underline{\hbox to 30mm{}} \\

    \end{tabular}
\end{center}

\section{基本积分方法}
\subsection{不定积分的凑微分求积分法}
    \textcolor{red}{\textbf{复习全书P89 \quad 二}}
\begin{multicols}{2}
$\int { f(ax+b)dx= } $
    \underline{\hbox to 30mm{}} \\
    %------------
    $\int{f(ax^2 + bx +c)(2ax + b)dx}=$
    \underline{\hbox to 30mm{}} \\
    %------------
    $\int{f(\ln{x})\frac{dx}{x}}=$
    \underline{\hbox to 30mm{}} \\
    %------------
    $\int{f(\sqrt{x})\frac{dx}{\sqrt{x}}dx}=$
    \underline{\hbox to 30mm{}} \\
    %------------
    $\int{f(\sin{x})\cos{x}dx}=$
    \underline{\hbox to 30mm{}} \\
    %------------
    $\int{f(\cos{x})\sin{x}dx}=$
    \underline{\hbox to 30mm{}} \\
    %------------
    $\int{f(\tan{x})\sec^2{x}dx}=$
    \underline{\hbox to 30mm{}} \\
    %------------
    $\int{f(\arcsin{x})\frac{dx}{\sqrt{1-x^2}}dx}=$
    \underline{\hbox to 30mm{}} \\
    %------------
    $\int{f(\arctan{x})\frac{dx}{1+x^2}}=$
    \underline{\hbox to 30mm{}} \\
    %------------
\end{multicols}
%---------------------- end document -------------------
\end{document}

